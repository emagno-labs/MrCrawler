\documentclass[twocolumn,showpacs,%
  nofootinbib,aps,superscriptaddress,%
  eqsecnum,prd,notitlepage,showkeys,10pt]{revtex4-1}

\usepackage[brazil]{babel}
\usepackage[utf8x]{inputenc}
\usepackage{amssymb}
\usepackage{amsmath}
\usepackage{graphicx}
\usepackage{dcolumn}
\usepackage{hyperref}

\begin{document}

\def\acknowledgmentsname{Agradecimentos}
\def\andname{e}
\def\indexname{Indíce}
\def\figuresname{Figuras}%
\def\tablename{TABELA}
\def\tablesname{Tabelas}%

\title{MR. CRAWLER: Auditoria de Vulnerabilidades de Aplicações Web}
\author{Eryckson Magno}
\affiliation{4º Período Bacharelado em Ciências da Computação (BCC) - Senac SP}
\author{Hamilton Santana}
\affiliation{4º Período Bacharelado em Ciências da Computação (BCC) - Senac SP}
\author{Ivan Probst}
\affiliation{4º Período Bacharelado em Ciências da Computação (BCC) - Senac SP}
\author{Lucas M. Ribeiro}
\affiliation{4º Período Bacharelado em Ciências da Computação (BCC) - Senac SP}

%\begin{abstract}
%em branco \dots
%\end{abstract}

\maketitle

\section{Introdução}
\label{sec:introducao}

É comum o uso de testes automatizados para averiguar o fator de segurança de aplicações web, ou seja, verificar sistematicamente vulnerabilidades nas mesmas.

As ferramentas que realizam esta tarefa são conhecidas como \emph{Web Application Security Scanners} (Scanners de Segurança de Aplicações Web). Para isto comumente são seguidas três etapas essenciais: 1) A ferramenta "varre" (\emph{crawl}) a aplicação web para relacionar todas as URLs e vetores de entrada, 2) Valores especiais para testes são submetidos para a aplicação web e 3) o scanner verifica os retornos recebidos procurando por evidências de vulnerabilidades.

Isto posto fica evidente a importância da sub-ferramenta que realiza a "varredura" (\emph{crawling}) da aplicação web. Tais sub-ferramentas (ou sub-sistemas), são conhecidos como \emph{Web Crawlers} (também como \emph{spiders} ou \emph{robots}) e representam um dos grandes desafios (se não o maior) de uma ferramenta de testes de segurança automatizados.

\subsection{Objetivo do Trabalho}

Este trabalho tem como objetivo apresentar uma ferramenta que analisa uma aplicação web e avalia seu grau de segurança (ou percentual de vulnerabilidade) fornecendo o resultado em uma visualização categorizada.

A ferramenta, nomeada \emph{MR. CRAWLER} no restante deste documento, é uma aplicação baseada em web (\emph{web-based application}) com código do lado servidor baseado em Python.

Suas funcionalidades podem ser sumarizadas em:

\begin{itemize}
\item Permitir ao usuário fornecer o endereço de um web site a ser analisado,
\item Processar a análise em três etapas: coleta (\emph{crawling}), avaliação (testes) e exibição de resultados e,
\item Permitir ao usuário que configure e compartilhe os tipos de visões sobre a análise realizada e os dados coletados: alterar agrupamentos e classificações, realizar \emph{drill-downs} e \emph{dril-ups} e exportar os dados.
\end{itemize}

Deve ser considerado o fato que ferramentas de análise automatizadas representam um certo grau de risco para aplicações web pois dependendo do tipo de análise que será realizado e da \emph{saúde} da aplicação a ser analisada, podem ser registrados cenários não desejados, chegando até ao extremo negativo de indisponibilizar a aplicação.

Devido a estes fatores o MR. CRAWLER dispõe de mecanismos que impedem que a análise realizada cause efeitos negativos no alvo:

\begin{itemize}
\item Limite de profundidade (links) de varredura,
\item Controle de tempo entre requisições,
\item Respeito a arquivos de exclusão (\emph{robots.txt}) e,
\item Não realização de \emph{stress tests} ou \emph{brute force}.
\end{itemize}

Porém tais mecanismos, quando acionados, irão eventualmente afetar a qualidade da análise. Assim sendo, em casos que se faça necessário o acionamento deste dispositivo, na apresentação de resultados o mesmo será evidenciado.

\subsection{Estado da Arte}

Segurança é um assunto que inevitavelmente fica em segundo plano em aplicações, em especial as voltadas para o mundo Web.

Em um ramo ditado pelas especificações funcionais e as regras de negócio, os \emph{donos} de projetos web quase sempre negligenciam ao aspecto \emph{segurança} em suas soluções e os desenvolvedores, pressionados e com pouco tempo para realizar as entregas, acabam seguindo o mesmo caminho.

Todavia, mesmo neste cenário desfavorável a concientização da indústria, do mercado e dos negócios sobre a importância da segurança cresce dia após dia (talvez, em parte, devido aos impactos negativos, e que geram prejuízos, de vazamentos e falhas de segurança apresentados pela mídia nos últimos tempos).

Atualmente encontram-se disponíveis no mercado várias ferramentas que, em diferentes níveis de especialização, confiabilidade e formato de distribuição (free, open-source ou comerciais), executam as tarefas de análise de segurança de aplicações web.

Diversos estudos avaliam e comparam os resultados destas ferramentas e um em especial \cite{metwas} relacionou outros estudos em uma análise comparativa.
 

%\subsubsection{\emph{Scanners} de Aplicações Web e sua evolução}

%\subsubsection{\emph{Web Crawlers} e como se relacionam com as vulnerabilidades}

%\subsubsection{Tendências Futuras}




\subsection{Vulnerabilidades de Aplicações Web}

em branco \dots

\subsubsection{Detecção de vulnerabilidades}

em branco \dots

\subsubsection{Desafios e Espectativas}

em branco \dots

\section{Desenvolvimento}
\label{sec:desenv}

em branco \dots

\subsection{Estratégias para auditoria}

em branco \dots

\subsubsection{Estratégias \emph{white-box}}

em branco \dots

\subsubsection{Estratégias \emph{black-box}}

em branco \dots

\section{Resultados}
\label{sec:resultados}

em branco \dots

\section{Conclusão}
\label{sec:conclusao}

em branco \dots


\begin{acknowledgments}

Os autores gostariam de agradecer:\cite{owasp2013} \dots

\end{acknowledgments}

\begin{thebibliography}{99}

\bibitem{owasp2013}
  \emph{The Open Web Application Security Project (OWASP)}
  http://www.owasp.org

\bibitem{metwas}
  MUÑOZ, Fernando Román, VILLALBA, Luis Javier García, METHODS TO TEST WEB APPLICATION SCANNERS
  
  
\end{thebibliography}

\end{document}
